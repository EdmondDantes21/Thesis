\chapter{Introduction}
\section{Context and motivation}
A team of researchers at the University of Trento developed a new general purpose BDI-based robotic system with ROS2. In synthesis this new framework allows programmers to automatically plan and execute actions which fulfill desires following the BDI (belief, desire, intention) philosophy, all whilst using the solid ROS2 communication techniques in the background. 
\par
For instance if one wanted to use this framework to automate robots which retrieve objects inside of a warehouse, all they would need to do is specify what the world looks like using a PDDL domain (what this is specifically will be explained later on); which actions can be performed and their definition (again, the particulars of this will be clear further on). The only thing that is left is to use ROS2 communication tools to continually update the framework on the state of the world.  
\par
The ROS2-BDI framework comes in two flavors: the offline version and the online version. Oversimplifying, the offline approach computes a plan and executes it until failure or success, rescheduling if and only if a failure occurs. The online version on the other hand initiates the execution as soon as it has plan, even if incomplete, then, during execution, if it observes any problems/opportunities it recomputes the plan with the goal of yielding a better solution. For instance, if the agent's goal is to pick up trash and it notices that there is some additional trash to pick up, it will try to come up with a plan to do just that.
\par
The framework is brand new, so much so that a solid validation has not yet been performed. Some attempts at it have been carried out by its creators, however, they do not always make use of industrially recognized tools such as Webots. A simulation was implemented using an ad hoc environment, which may not always be representative of a real world scenario. Conversely, Webots, which is a 3D graphical robot simulator, allows to have a better view of what a real world implementation could look like. 
\section{Problem overview}
In light of all this the first part of the project's objective was dual: firstly, it would try to demonstrate the effectiveness of the framework on Webots; secondly, it tries to show that in highly dynamic environments the online approach bests the offline system. This was accomplished by creating multiple situations where the goal can be quantifiable and hence maximized.
\par
The second part of the project was dedicated to studying the flexibility of the framework. In other words it was tried to prove that once the initial setup is completed, it is pretty straightforward to modify or update the code to fulfill the new objective. This was done by analyzing the differences in code when different kind of changes to the environment or objective were made.
\section{Accomplished results}
The most noteworthy accomplishment is the set of simulations that were created, which not only serve the purpose to validate the efficacy of the ROS2-BDI robotic system, but also as a reference for future researchers/developers who will have the pleasure to work with the framework. The code was written following widely known software engineering guidelines at the best of the abilities of the author, hopefully that makes the code readable and easy to follow as an implementation example. To clarify, what is trying to be said is that the code could be consulted to have a good idea of how to interface with the framework and how to use it in real world scenarios. 

